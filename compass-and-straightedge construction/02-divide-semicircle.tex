\documentclass[tikz, border=5pt]{standalone}
% 用于计算点的坐标
\usetikzlibrary{calc}
% 用于计算交点
\usetikzlibrary{intersections,through}

\begin{document}
\begin{tikzpicture}

    % \draw[step=1cm,very thin, gray] (-2,0) grid (3,-4);
    \coordinate[label=above:$A$] (A) at (1,0);
    \coordinate[label=below:$B$] (B) at (1,-4);

    % 绘制并标记半圆为H
    \draw (A) arc (90:270:2) coordinate [name path=H,circle through=(A)] (H) at ($(A)!.5!(B)$);

    % 绘制线段AB
    \path [draw] (A) -- (B);
    % 以A,B为圆心, AB长为半径做圆弧
    \draw let \p1=($ (A) - (B) $), \n2={veclen(\x1, \y1)}
          in  (A) arc (90:20:\n2)
              (B) arc (-90:-20:\n2);
    % 标记两个圆弧段的位置
    \node (D) [name path=D,circle through=(B)] at (A) {}; 
    \node (E) [name path=E,circle through=(A)] at (B) {};

    % 标记第二个交点为O 
    \path [name intersections={of=D and E}] 
        coordinate [label=right:$O$] (O) at (intersection-2);

    % 循环绘制各交点: 这里需要分成两部分进行绘制
    \foreach \i in {1,2}{
        % 等分直径AB并做标记
        \coordinate[label=above right:$\i$] (\i) at ($(A)!\i/5!(B)$);
        % 作延长线(不绘制)
        \path [name path=O--\i] (O) -- ($(O)!2!(\i)$);
        % 标记交点
        \path [name intersections={of=H and O--\i,by={\i'}}] 
          coordinate[label=above left:$\i'$] (\i') at (intersection-2);
        \draw (O) -- (\i');  
}
    \foreach \i in {3,4}{
        % 等分直径AB并做标记
        \coordinate[label=below right:$\i$] (\i) at ($(A)!\i/5!(B)$);
        % 作延长线(不绘制)
        \path [name path=O--\i] (O) -- ($(O)!2!(\i)$);
        % 标记交点
        \path [name intersections={of=H and O--\i,by={\i'}}] 
          coordinate[label=below left:$\i'$] (\i') at (intersection-1);
        \draw (O) -- (\i');  
}
\end{tikzpicture}
\end{document}
