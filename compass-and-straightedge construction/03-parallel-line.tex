\documentclass[tikz, border=5pt]{standalone}
% 用于计算点的坐标
\usetikzlibrary{calc}
% 用于计算交点
\usetikzlibrary{intersections,through}

\begin{document}
\begin{tikzpicture}
    % \draw[help lines] (-2,-2) grid (5,5);
    % 绘制直线l
    \coordinate [] (M) at (-.5,0); 
    \coordinate [label=right:{$\ell$}] (N) at (3.5,0);
    \draw[name path=l] (M) -- (N);

    % 标记直线外一点A
    \coordinate [label=above:$A$] (A) at (.5,1);
    \fill (A) circle (1pt);

    % 以大于$A$到$l$距离的长度为半径,以$A$为圆心画圆弧$r$,交直线$l$于$B$点.
    \coordinate [label=above right:$D$] (D) at ($(A)+(1.25,0)$);
    \draw[red] (D) arc (0:-60:1.25)
               (D) arc (0:10:1.25);
    \node[below right] (P) at (1.5,.5) {$r$};
    \coordinate [name path=r,circle through=(D)] (r) at (A);

    % 标记交点B
    \path [name intersections={of=r and l}] 
        coordinate [label=below left:$B$] (B) at (intersection-2);
    \fill (B) circle (1pt);

    % 以$B$为圆心,相同半径画圆弧,在相同方向上交直线$l$ 于$C$点.
    \coordinate[label=below right:$C$] (C) at ($(B)+(1.25,0)$);
    \draw [red] (C) arc (0:10:1.25)
                (C) arc (0:-10:1.25);
    \fill (C) circle (1pt);

    % 以$C$为圆心,相同半径画圆弧,交圆弧$r$于$D$点(非$B$点), 连接$AD$,则所作直线$AD$就是所求的直线.
    \draw [red] (D) arc (127:137:1.25)
                (D) arc (127:117:1.25);
    \fill (D) circle (1pt);
    \draw ($(A)!-.7!(D)$) -- ($(A)!1.75!(D)$);
\end{tikzpicture}
\end{document}
